\documentclass{article}
\usepackage{main}

% Aquí empieza el documento{{{
\begin{document}

%\maketitle
\thispagestyle{fancy}

\begin{figure}[H]
	\centering
	\begin{tikzpicture}
	[
		circuit ee IEC,
		set resistor graphic=var resistor IEC graphic
	]
	\boldmath

	\draw (0,0)
		to[battery={pos=0.3, info'=$9V$}]
		++(6,0);

	\draw (4,0)
		to[resistor={info'=$1k\Omega$}]
		++(0,2);

	\draw (6,0)
		to[resistor={info'=$1k\Omega$}]
		++(0,2);

	\draw (0,2)
		to
		[
			resistor={pos=0.4, info=$15k\Omega$},
			resistor={pos=((4+6)/12), info=$\cancel{33\Omega}$}
		]
		++(6,0);

	\draw (4,2) -- (6,4);

	\draw (6,2) -- ++(0,2)
		to
		[
			resistor={pos=0.25, info'=$10k\Omega$},
			resistor={pos=0.75, info'=$2k\Omega$}
		]
		++(-6,0)
		to[resistor={info'=$2k\Omega$}]
		++(0,-2)
		-- (0,0);

	\end{tikzpicture}
\end{figure}


\begin{figure}[H]
	\centering
	\begin{tikzpicture}
	[
		circuit ee IEC,
		set resistor graphic=var resistor IEC graphic
	]
	\boldmath

	\draw (0,0)
		to[battery={pos=0.3, info'=$9V$}]
		++(6,0);

	\draw (4,0)
		to[resistor={info'=$1k\Omega$}]
		++(0,2);

	\draw (6,0)
		to[resistor={info'=$1k\Omega$}]
		++(0,2);

	\draw (0,2)
		to
		[
			resistor={pos=0.4, info=$15k\Omega$},
			%resistor={pos=((4+6)/12), info=$\cancel{33\Omega}$}
		]
		++(6,0);

	%\draw (4,2) -- (6,4);

	\draw (6,2) -- ++(0,2)
		to
		[
			resistor={pos=0.25, info'=$10k\Omega$},
			resistor={pos=0.75, info'=$2k\Omega$}
		]
		++(-6,0)
		to[resistor={info'=$2k\Omega$}]
		++(0,-2)
		-- (0,0);

	\end{tikzpicture}
\end{figure}

\begin{table}[H]
	\centering
	\begin{tabular}{|c|c|c|c|}
		\hline
		Mediciones o lecturas de instrumentos & Valor teórico &
		Valor expermiental & Porcentaje de error\\
		\hline
		Lectura del amperímetro $A_1$\\
		\hline
		Corriente en la resistencia de $5k\Omega$ &
		\multicolumn{3}{c|}{No existe}\\
		\hline
		Lectura del amperímetro $A_2$\\
		\hline
		Corriente en la resistencia de $10k\Omega$\\
		\hline
		Lectura del amperímetro $A_3$\\
		\hline
		Voltaje en la resistencia de $30k\Omega$ &
		\multicolumn{3}{c|}{No existe}\\
		\hline
		Lectura del voltímetro $V_1$\\
		\hline
		Lectura del voltímetro $V_2$\\
		\hline
	\end{tabular}
\end{table}

\end{document}
%}}}
